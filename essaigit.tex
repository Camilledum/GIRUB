
\documentclass[12pt]{article}

\usepackage[utf8]{inputenc}
\usepackage[T1]{fontenc}
\usepackage{times}
\usepackage{natbib}
\usepackage[francais]{babel}
\usepackage{physics}
\usepackage{amsmath, amssymb}
\usepackage{graphicx}

\usepackage[left=2cm, right=2cm, bottom=2cm]{geometry}
\title{Déterminants de la mobilité frontalière du travail: comparaison entre Suisse et Luxembourg}
\author{Camille Dumeignil}
\date{\today}
\begin{document}
\maketitle
\tableofcontents % créer une table des matières 


\section{Introduction}

En 2011, plus de 350 000 personnes travaillent dans un pays voisin tout en résidant en France, c’est 50 000 de plus qu’en 2006 et 100 000 de plus qu’en 1999 soit une progression de 40\% entre 1999 et 2011 (INSEE). Ce nombre est en constante augmentation chaque année ce qui témoigne d’un attrait croissant pour le travail frontalier. L’Union Européenne définit le travailleur frontalier comme « tout travailleur qui est occupé sur le territoire d’un état membre et qui réside sur le territoire d’un autre état membre, où il retourne en principe chaque jour ou au moins une fois par semaine ». En plus d’avoir une croissance du phénomène, les distances entre le domicile et le travail ont augmenté considérablement dans les économies développées \cite{Rouwendal1999}. Il est donc possible d’avoir une mobilité au travail qui ne s’accompagne pas d’une mobilité géographique, ce sont les migrations alternantes. Elles se définissent comme des déplacements journaliers entre lieu de travail et lieu d’habitation (\cite{Mougenot1985}). En France, on dénombre huit destinations pour le travail frontalier (Allemagne, Belgique, Espagne, Italie, Luxembourg, Monaco, Royaume-Uni, Suisse).

On constate bien à travers la figure 1 l’explosion du nombre de travailleurs frontaliers en France. Il nous renseigne aussi sur les destinations privilégiées des Français à savoir la Suisse en première position, le Luxembourg et l’Allemagne qui engrangent la majorité des flux de travailleurs. Par exemple, la Suisse accueillait près de 250 000 frontaliers en 2012, dont 21\% dans le canton de Bâle, 21\% dans le canton de Tessin et parmi-eux 53 proviennent de France. Dans le canton de Genève, les Français sont surreprésentés, pour atteindre 93\%  des frontaliers (Delaugerre,2012). 

Pour \cite{Mougenot1985}, les mobilités alternantes résultent d’un choix individuel. Les personnes comparent ce qui est possible ou non en faisant un calcul rationnel sur trois éléments essentiels à prendre en considération à savoir le temps, l’espace et le coût de déplacement, qu’il soit monétaire ou psychologique. Ils sont mis en relation avec les bénéfices du travail. La perception de la migration alternante est donc induite par le rapport que les individus entretiennent avec ces facteurs (\cite{Mougenot1985}). 

Au niveau international, une partie de la littérature étudie le phénomène mais dans d’autres pays tels que les Pays-Bas ou les Etats-Unis. Cela permet d’avoir un cadre théorique de référence. Beaucoup d’articles s’intéressent à la mobilité du travail entre une zone urbaine et une zone rurale. Ici, il conviendrait de spécifier cette mobilité au niveau de deux pays qui sont très proches géographiquement.

Les migrations alternantes entre le Luxembourg et la France sont les plus étudiées dans la littérature, mais il n’y a aucune comparaison entre les différentes zones dans la même étude. La littérature ne permet pas de saisir l’hétérogénéité entre les différents pays c’est pourquoi nous souhaitons nous intéresser aux déterminants de la mobilité frontalière du travail en menant une étude comparative entre la Suisse et le Luxembourg. Les populations concernées par ce phénomène peuvent avoir des caractéristiques différentes en fonction du pays dans lequel elles vont travailler. Il est important aussi de regarder les différences entre un travailleur frontalier et un non-frontalier pour saisir l’hétérogénéité individuelle. L’objectif de traiter cette thématique est de comprendre, par la suite, les impacts du mouvement de main d’œuvre sur le territoire Français.

La littérature nécessaire pour répondre à cette problématique est assez diversifiée car les cadres analytiques ne sont que peu adaptés à l’étude des mobilités transfrontalières du travail, puisque les mobilités ne sont pas complètes. Les individus sont mobiles du point de vue du travail, mais restent habiter sur le territoire de départ. Il faut regarder les recherches en économie du travail, en économie géographique mais aussi  en économie de la famille pour élaborer une étude complète sur le sujet car chaque élément est susceptible d’influencer la probabilité d’être un travailleur frontalier.
Pour répondre à cette problématique il convient dans un premier temps d’étudier les déterminants théoriques de la mobilité au travail comme le contexte institutionnel et les éléments géographiques qui peuvent favoriser la circulation des personnes entre deux pays. Cependant, à la lumière de la littérature sur le sujet, ces deux éléments ne constituent pas les seuls déterminants du travail frontalier. Plusieurs auteurs (\cite{Levy1985} \cite{White1986}  ; \cite{Rouwendal1999}; \cite{Gorter1997} ; \cite{Lee2003} ; \cite{Mathias2003} ; \cite{Corvers2003}) mettent en avant l’importance des caractéristiques individuelles et de l’arbitrage coût / bénéfice qui est fait concernant la durée de déplacement journalier (\cite{Mougenot1985}). Il est donc nécessaire de prendre en compte l’ensemble des critères institutionnels, géographiques mais aussi individuels pour bien comprendre la migration frontalière. Chaque élément pris à son niveau peut influencer le flux de main d’œuvre traversant chaque jour les frontières.

Dans un second temps, il faudra tester les hypothèses théoriques mises en avant grâce à la revue de littérature. Pour ce faire nous allons utiliser l’enquête du recensement de la population de 2012 réalisé par l’Insee et plus précisément le fichier détail sur la mobilité professionnelle. D’abord, nous allons présenter l’enquête et faire une analyse descriptive de l’ensemble des variables qui auront été sélectionnées pour répondre à la problématique et en fonction des éléments disponibles dans la base de données. Ensuite, nous testerons les différents postulats mis en avant dans la littérature à l’aide d’une régression économétrique

\section{Partie 1: Déterminants théoriques}
\subsection{Eléments géographiques}
\subsection{Dispositifs institutionnels}
\subsection{Caractéristiques du marché du travail}
\subsection{Déterminants individuels}
\section{Modèles économétriques}
\subsection{Données et statistiques descriptives}
\subsection{Méthodes}
\subsection{Résultats}

\bibliographystyle{agsm}
\bibliography{D:/Doctorat/Biblio/Bibthese}


\end{document}
\end{large}
\end{normalsize}